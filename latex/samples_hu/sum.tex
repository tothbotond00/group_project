\chapter{Összegzés}
\label{ch:sum}

A tevékenységmodul számos eddig Moodle által nem támogatott funkcióval teszi könnyebbé a tanárok számára a csoportos feladatok megszervezését. Habár a keretrendszerben van lehetőség feladatot Moodle által karban tartott csoportok számára kiadni, ez csak egy később implementált kiegészítés volt a feladattípusú tevékenységmodulhoz. A szakdolgozatban bemutatott segédprogram számos igényt kielégít egy modulban, ezzel izoláltan kezeli az adatokat.

A szakdolgozatban bemutatott segédprogram mintegy mintaként is szolgálhat későbbi Moodle fejlesztések számára. Az alkalmazás fejlesztése során külön figyelmet kaptak olyan megoldások, amelyek megoldása egy teljesen más stack-kel vagy technológiával sikerült volna. Az ilyen problémákra mindig volt egy Moodle által támogatott megoldás/javaslat, mellyel a problémát át lehett hidalni. A fejlesztés során az összes egyedi problémára a Moodle által támogatott megoldás került implementálásra.

A Moodle hatalmas szabadságot ad a rendszerén belül. Érthető és konzekvens API-kat biztosít a fejlesztők számára. Az alaprendszerbe integrált segédprogramok is Moodle szabvány szerint kerültek implementálásra, így a forráskód birtokában egy szinte hivatalos dokumentációt tudunk értelmezni. Jelen segédprogram bővítése sem jelentene kihívást egy tapasztalt Moodle fejlesztőnek, mivel a Moodle rendszerén belül lefektetett konvencióknak megfelel.
