\chapter{Bevezetés}
\label{ch:intro}

A Moodle napjaink egyik legelterjedtebb nyílt forráskódú elearning keretrendszere. A rendszer nagy előnye, hogy nagyon egyszerűen egészíthető ki segédprogramokkal, így egyedi igényekkel tudjuk a rendszer komponenseit bővíteni anélkül, hogy az eredeti forráskódot módosítanunk kellene. A rendszer lehetőség ad egyedi tevékenységmodulok, beiratkozási lehetőségek vagy akár riportok fejlesztésére. A Moodle-ben minden tevékenységet, minden felhasználó egyedileg old meg, aktív csoportos feladatmegoldásra nincs lehetőség. Szakdolgozatom témája egy egyedi tevékenységmodul\footnote{Tevékenységmodul: A Moodle kurzusainak alkotóelme. Minden kurzus tevékenységmodulokból épül fel, melyeknek különböző tulajdonságai vannak (például Scorm csomag, H5P modul, Kérdőív) } , melyben csoportos feladatmegoldásra van lehetősége a hallgatóknak. 

A modul célja, hogy egy tevékenységen belül valósuljon meg a csoportok és szerepek kiosztása, a csoportokon belüli csevegés, feladat beadása, illetve a feladat értékelése minden csoporttag számára. A segédprogram támogatni fogja a Moodle-ben definiált szerepek használatát, de saját egyedi szerepeket is kioszthatunk a csoportunkon belül.

Ezeken felül a fejlesztés célja, hogy a modul feleljen meg a Moodle keretrendszer konvencióinak és aktívan használja az LMS\footnote{LMS: Learning Management System, elearning keretrendszer} által nyújtott lehetőségeket. Fejlesztésemben nagy hangsúly került a Moodle- integrációra, így számos a rendszer által támogatott plusz funkciót támogat a segédprogram. A fejlesztés alatt egység- és eseményvezérelt tesztelést használtam.
